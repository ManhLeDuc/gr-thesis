\documentclass[../DoAn.tex]{subfiles}
\begin{document}

\section{Đặt vấn đề}
\label{section:1.1}
Công nghệ chuỗi khối (blockchain) được cho là một trong những công nghệ cốt lỗi trong cuộc cách mạng công nghiệp lần thứ 4. Tính đột phá của công nghệ chuỗi khối nằm ở việc cho cho phép những giao dịch có giá trị lớn được thực hiện một cách minh bạch, chính xác, công bằng mà không cần đến sự chứng thực và phân xử của một bên thứ ba. Cốt lõi của công nghệ chuỗi khối nằm ở một cuốn sổ cái kỹ thuật số và các ứng dụng phi tập trung. Cuốn sổ cái cơ bản là một chuỗi các khối dữ liệu (nên có tên là chuỗi khối) được lưu trữ phân tán trên nhiều máy. Dữ liệu trong cuốn sổ cái này là không thể giả mạo và một khi đã được thêm mới vào thì sẽ không thể thay đổi hay xóa bỏ. Ứng dụng phi tập là một đoạn mã được lập trình để thực hiện thêm mới dữ liệu vào sổ cái. Đoạn mã này có thể được thực thi một cách tự động với tính chính xác cao. Dữ liệu được cập nhật bởi ứng dụng phi tập trung sẽ đảm bảo tính minh bạch cao cùng khả năng truy xuất nhờ vào đặc tính không thể bị can thiệp hay sửa đổi của sổ cái.

Những mạng chuỗi khối nổi tiếng nhất như Bitcoin và Ethereum là mạng chuỗi khối công khai (public blockchain). Đối với loại mạng này, mọi người đều có thể tham gia, thực hiện tương tác hay truy vấn dữ liệu. Tuy nhiên, đối với một số nghiệp vụ, việc thông tin giao dịch và số liệu hoạt động của các doanh nghiệp bị công khai có thể là một vấn đề lớn. Do vậy việc sử dụng loại mạng công khai vào trong nghiệp vụ giữa các tổ chức, liên doanh đôi khi là bất khả thi. Để giải quyết vấn đề này mạng chuỗi khối riêng tư (private blockchain) ra đời. Để có thể được ứng dụng vào các hoạt động giữa các doanh nghiệp, ngoài tính phân tán, bảo mật và minh bạch dữ liệu, mạng riêng tư thường thêm có các đặc điểm sau: (i) Danh tính của những bên tham gia vào mạng cần phải xác thực được.(ii) Không phải ai cũng có thể tham gia, truy vấn dữ liệu hay tương tác với mạng, chỉ các cá nhân, tổ chức có đủ quyền mới có thể thực hiện các hành động này. (iii) Tốc độ xử lý giao dịch cao hơn nhiều so với mạng chuỗi khối công khai.

Với những ưu điểm mà mạng chuỗi khối riêng tư mang lại, việc các doanh nghiệp ứng dụng mạng riêng tư vào nghiệp vụ chắc chắn có thể giúp nâng cao hiệu suất công việc. Tuy vậy quá trình ứng dụng này thường gặp phải các khó khăn sau. Đầu tiên, do là mạng riêng tư, cơ sở hạ tầng mạng sẽ cần phải được quản lý riêng biệt. Mỗi một mạng chuỗi khối sẽ có một kiến trúc riêng. Doanh nghiệp muốn sử dụng sẽ cần nghiên cứu kiến trúc đó rồi triển khai một mạng lên một cơ sở hạ tầng cụ thể. Không chỉ vậy, cơ sở hạ tầng cũng cần được cân nhắc để đảm bảo tính ổn định và khả năng chịu lỗi để có thể được sử dụng lâu dài. Cũng giống như kiến trúc mạng, ứng dụng phi tập trung trên mạng chuỗi khối riêng tư cũng có một cấu trúc và cách thức hoạt động riêng biệt. Do vậy việc phát triển và ứng dụng các ứng dụng phi tập trung phục vụ cho nghiệp vụ cũng là một thử thách. Nhà phát triển cần tìm hiểu cách hoạt động rồi từ cách hoạt động đó phát triển một ứng dụng mới đáp ứng được nhu cầu nghiệp vụ. Thử thách này còn trở nên đặc biệt khó khăn đối với những ai chưa từng tiếp xúc với công nghệ chuỗi khối.

Có thể thấy, để ứng dụng những đặc tính nổi bật của mạng chuỗi khối riêng tư vào các nghiệp vụ thực tiễn yêu cầu những kiến thức đặc thù cùng nhiều nhân lực và thời gian. Nhận thấy vấn đề này, tôi đã quyết định phát triển một hệ thống triển khai mạng cùng với đó là ứng dụng phi tập trung dựa trên nền tảng Hyperledger Fabric - một nền tảng mạng chuỗi khối riêng tư rất phổ biến hiện nay. Thông qua giao diện trực quan của hệ thống, một lập trình viên dù cho không có kiến thức về mạng chuỗi khối cũng có thể dễ dàng triển khai hạ tầng mạng và các ứng dụng phi tập trung phục vụ cho nhiều yêu cầu nghiệp vụ khác nhau.

\section{Mục tiêu và phạm vi đề tài}
\label{section:1.2}

\subsection{Giải pháp liên quan}
Ông lớn Amazon cung cấp một giải pháp để hỗ trợ quá trình triển khai mạng Hyperledger Fabric - Amazon Managed Blockchain\cite{amazon}. Với thế mạnh về điện toán đám mây của mình, thông qua dịch vụ Managed Blockchain, người dùng có thể triển khai một mạng Hyperledger Fabric cho riêng mình trên cơ sở hạ tầng của Amazon chỉ với vài cái nhấp chuột. Quá trình quản lý và theo dõi hoạt động của mạng cũng được đơn giản hóa, tối ưu trải nhiệm người dùng. 

Tuy quy trình khởi tạo và quản trị mạng có thể được thực hiện thông qua giao diện trực quan, việc phát triển ứng dụng phi tập trung phục vụ cho các nghiệp vụ trên mạng này lại không được như vậy. Người dùng phải tự mình lập trình và chạy các câu lệnh trên terminal để triển khai ứng dụng đó lên mạng. Do vậy quá trình phát triển ứng dụng phi tập trung vẫn sẽ yêu cầu kiến thức chuyên sâu về Hyperledger Fabric. Lập trình viên vẫn sẽ cần phải nghiên cứu để viết và triển khai hợp đồng thông minh, một quá trình tiêu tốn nhiều thời gian lẫn công sức.

\subsection{Mục tiêu và phạm vi}

Với mục tiêu đơn giản hóa nhất có thể quá trình ứng dụng mạng chuỗi khối riêng tư vào các nghiệp vụ doanh nghiệp, hệ thống trong đồ án này hướng đến việc cho phép những lập trình viên dù cho không có kiến thức về mạng chuỗi khối cũng có thể dễ dàng triển khai mạng cùng với đó là ứng dụng phi tập trung dựa trên nền tảng Hyperledger Fabric. Để đạt được điều này, 2 mục tiêu sau được đề ra:
\begin{itemize}
	\item Cho phép cấu hình và triển khai một mạng Hyperledger Fabric thông qua giao diện trực quan.
	\item Cho phép thiết kế và triển khai ứng dụng phi tập trung thông qua giao diện trực quan. Cung cấp phương thức để có thể tương tác với hợp đồng thông minh đó mà không yêu cầu kiến thức đặc thù.
\end{itemize}

\section{Định hướng giải pháp}
\label{section:1.3}

Để giải quyết vấn đề đã nều ở mục 1.1, hệ thống sẽ hỗ trợ hai chức năng chính cho người dùng.

\subsection{Triển khai mạng}
Cơ sở hạ tầng để triển khai mạng cần được suy tính kỹ lượng để đảm bảo mạng hoạt động ổn định và hiệu quả nhất. Để đơn giản hóa quá trình triển khai và quản lý mạng Hyperledger Fabric, người dùng sẽ được phép tùy chỉnh cấu hình thông qua một giao diện. Sau đó, hệ thống sẽ triển khai một mạng với cấu hình tương ứng lên điện toán đám mây. Ngươi dùng sẽ không cần chú ý quá nhiều đến chi tiết phần cứng mà chỉ cần quan tâm đến cấu hình của cơ sở hạ tâng. 
\subsection{Triển khai ứng dụng phi tập trung}
Điều quan trọng nhất trong việc triển khai ứng dụng là thiết kế kiến trúc sao cho phù hợp với nghiệp vụ. Để bao quát được nhiều nghiệp vụ nhất có thể, hệ thống sẽ cho phép người dùng triển khai ứng dụng dựa trên mô hình cơ sở dữ liệu quan hệ. Nhiều thực thể sở hữu các thuộc tính khác nhau có thể được định nghĩa. Các thực thể này có thể có nhiều liên kết với nhau (quan hệ một-một, một-nhiều, nhiều-nhiều). Dựa vào kiến trúc tổng quan của các thực thể này, một ứng dụng phi tập trung tương ứng sẽ được tự động sinh ra. Sau khi được triển khai, người dùng có thể tải một bộ SDK được hệ thống sinh ra về để tương tác với mạng chuỗi khối thông qua các hàm đọc ghi sửa xóa các thực thể trên. Quá trình hình thành và thay đổi của các thực thể này sẽ được lưu lại vĩnh viễn, đảm bảo việc xác thực và truy vấn dữ liệu nghiệp vụ thông qua mạng chuỗi khối.

\section{Bố cục đồ án}
\label{section:1.4}
Phần còn lại của báo cáo đồ án tốt nghiệp này được tổ chức như sau:

\begin{itemize}
	\item Chương 2: Giới thiệu kiến trúc mạng chuỗi khối Hyperledger Fabric.
	\item Chương 3: Khảo sát và phân tích yêu cầu.
	\item Chương 4: Trình bày về các công nghệ sử dụng.
	\item Chương 5: Trình bày về xây dựng và đánh giá hệ thống.
	\item Chương 6: Kết luận và định hương phát triển trong tương lai.
\end{itemize}

% Chương 2 trình bày về kiến trúc mạng chuỗi khối Hyperledger Fabric.

% Trong Chương 3, em/tôi giới thiệu về v.v.

% \textbf{Chú ý:} Sinh viên cần viết mô tả thành đoạn văn đầy đủ về nội dung chương. Tuyệt đối không viết ý hay gạch đầu dòng. Chương 1 không cần mô tả trong phần này. 

% Ví dụ tham khảo mô tả chương trong phần bố cục đồ án tốt nghiệp: Chương *** trình bày đóng góp chính của đồ án, đó là một nền tảng ABC cho phép khai phá và tích hợp nhiều nguồn dữ liệu, trong đó mỗi nguồn dữ liệu lại có định dạng đặc thù riêng. Nền tảng ABC được phát triển dựa trên khái niệm DEF, là các module ngữ nghĩa trợ giúp người dùng tìm kiếm, tích hợp và hiển thị trực quan dữ liệu theo mô hình cộng tác và mô hình phân tán.

% \textbf{Chú ý:} Trong phần nội dung chính, mỗi chương của đồ án nên có phần Tổng quan và Kết chương. Hai phần này đều có định dạng văn bản “Normal”, sinh viên không cần tạo định dạng riêng, ví dụ như không in đậm/in nghiêng, không đóng khung, v.v. 

% Trong phần Tổng quan của chương N, sinh viên nên có sự liên kết với chương N-1 rồi trình bày sơ qua lý do có mặt của chương N và sự cần thiết của chương này trong đồ án. Sau đó giới thiệu những vấn đề sẽ trình bày trong chương này là gì, trong các đề mục lớn nào.

% Ví dụ về phần Tổng quan: Chương 3 đã thảo luận về nguồn gốc ra đời, cơ sở lý thuyết và các nhiệm vụ chính của bài toán tích hợp dữ liệu. Chương 4 này sẽ trình bày chi tiết các công cụ tích hợp dữ liệu theo hướng tiếp cận “mashup”. Với mục đích và phạm vi của đề tài, sáu nhóm công cụ tích hợp dữ liệu chính được trình bày bao gồm: (i) nhóm công cụ ABC trong phần 4.1, (ii) nhóm công cụ DEF trong phần 4.2, nhóm công cụ GHK trong phần 4.3, v.v.

% Trong phần Kết chương, sinh viên đưa ra một số kết luận quan trọng của chương. Những vấn đề mở ra trong Tổng quan cần được tóm tắt lại nội dung và cách giải quyết/thực hiện như thế nào. Sinh viên lưu ý không viết Kết chương giống hệt Tổng quan. Sau khi đọc phần Kết chương, người đọc sẽ nắm được sơ bộ nội dung và giải pháp cho các vấn đề đã trình bày trong chương. Trong Kết chương, Sinh viên nên có thêm câu liên kết tới chương tiếp theo.

% Ví dụ về phần Kết chương: Chương này đã phân tích chi tiết sáu nhóm công cụ tích hợp dữ liệu. Nhóm công cụ ABC và DEF thích hợp với những bài toán tích hợp dữ liệu phạm vi nhỏ. Trong khi đó, nhóm công cụ GHK lại chứng tỏ thế mạnh của mình với những bài toán cần độ chính xác cao, v.v. Từ kết quả nghiên cứu và phân tích về sáu nhóm công cụ tích hợp dữ liệu này, tôi đã thực hiện phát triển phần mềm tự động bóc tách và tích hợp dữ liệu sử dụng nhóm công cụ GHK. Phần này được trình bày trong chương tiếp theo – Chương 5.

\end{document}
