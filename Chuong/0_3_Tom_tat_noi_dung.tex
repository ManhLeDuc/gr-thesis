\documentclass[../DoAn.tex]{subfiles}
\begin{document}

\begin{center}
    \Large{\textbf{TÓM TẮT NỘI DUNG ĐỒ ÁN}}\\
\end{center}
\vspace{1cm}

Công nghệ chuỗi khối nổi lên gần đây với khả năng cho phép các giao dịch có giá trị lớn và quan trọng có thể được thực hiện một cách an toàn và chính xác mà không yêu cầu có sự xuất hiện và can thiệp của một bên thứ ba trung gian. Với lợi thế này, việc ứng dụng công nghệ chuỗi khối vào thực tiễn sẽ giúp nâng cao cải thiện hiệu suất công việc cũng như tiết kiệm chi phí vận hành cho các nghiệp vụ. Thế nhưng do là một công nghệ mới và có độ phức tạp cao, việc ứng dụng công nghệ chuỗi khối vào các nghiệp vụ doanh nghiệp là một quá trình yêu cầu nhiều công sức và thời gian. Mục tiêu của đồ án này là phát triển một hệ thống hỗ trợ đơn giản hóa nhất có thể quá trình ứng dụng này thông qua cơ chế tự động triển khai mạng và ứng dụng phi tập trung dựa trên nền tảng Hyperledger Fabric - một nền tảng chuỗi khối mạnh mẽ sở hữu nhiều tính năng phù hợp với các hoạt động kinh doanh.

"Hệ thống triển khai mạng và ứng dụng phi tập trung dựa trên nền tảng Hyperledger Fabric" cho phép người dùng xây dựng và phát triển một cơ sở hạ tầng mạng và các ứng dụng phi tập trung để ứng dụng trong nghiệp vụ thông qua các thông số và cấu hình tổng quan. Hạ tầng mạng sẽ được triển khai trên một nền tảng Kubernetes để đảm bảo tính ổn định cùng với khả năng mở rộng dễ dàng. Từ một mô hình cơ sở dữ liệu quan hệ, một ứng dụng phi tập trung cũng sẽ được tự động viết và triển khai lên mạng. Ngoài ra việc tương tác với ứng dụng được đơn giản hóa thông qua một bộ SDK được tạo ra bởi hệ thống.

Hệ thống được triển khai theo kiến trúc Microservices với nhiều dịch vụ nhỏ nhằm đảm bảo khả năng mở rộng. Các dịch vụ này được phát triển sử dụng ngôn ngữ lập trình Python và giao diện sử dụng thư viện ReactJS. Đề tài V-chain\cite{vchain} đã sử dụng hệ thống trong đồ án này như một dịch vụ xử lý các tác vụ liên quan đến mạng và ứng dụng thuộc nền tảng Hyperledger Fabric.

\end{document}
